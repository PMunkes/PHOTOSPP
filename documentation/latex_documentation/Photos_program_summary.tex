{\bf PROGRAM SUMMARY}

\begin{small}
\noindent
{\em Manuscript Title:}                                       \\
PHOTOS Interface in C++; Technical and Physics Documentation \\
{\em Authors:}                                                \\
N. Davidson, T. Przedzinski, Z. Was \\
{\em Program Title:}                                          \\
Photos++ \\
{\em Journal Reference:}                                      \\
  %Leave blank, supplied by Elsevier.
{\em Catalogue identifier:}                                   \\
  %Leave blank, supplied by Elsevier.
{\em Licensing provisions:}                                   \\
  %enter "none" if CPC non-profit use license is sufficient.
none. {\tt HepMC} event record interface, built as separate
library {\tt libPhotosHepMC.so}, is licensed under GNU GPL.   \\
{\em Programming language:}                                   \\
C++ \\
{\em Computer:}                                               \\
  %Computer(s) for which program has been designed.
PC \\
{\em Operating system:}                                       \\
  %Operating system(s) for which program has been designed.
Linux, MacOS \\
{\em RAM:} bytes                                              \\
  %RAM in bytes required to execute program with typical data.
Libraries take less than 2MB. Memory complexity is $O(n)$ with around 2-4MB for
events with 10k particles. \\
{\em Number of processors used:}                              \\
  %If more than one processor.
{\em Supplementary material:}                                 \\
  % Fill in if necessary, otherwise leave out.
{\em Keywords:} Keyword one, Keyword two, Keyword three, etc.  \\
  % Please give some freely chosen keywords that we can use in a
  % cumulative keyword index.
PHOTOS, QED, Bremsstrahlung, FSR, photon emission in decays, final state radiation \\
{\em Classification:}                                         \\
  %Classify using CPC Program Library Subject Index, see (
  % http://cpc.cs.qub.ac.uk/subjectIndex/SUBJECT_index.html)
  %e.g. 4.4 Feynman diagrams, 5 Computer Algebra.
100: Physics of Elementary Particles and Fields \\
{\em External routines/libraries:}                            \\
  % Fill in if necessary, otherwise leave out.
{\em Subprograms used:}                                       \\
  %Fill in if necessary, otherwise leave out.
{\em Nature of problem:}\\
  %Describe the nature of the problem here.
Algorithm described in this paper can be used to add final state radiation
to the event generated by external software using selected event record format.
It can also be used on a sample of events loaded from data file.
User can define parts of the decay tree on which algorithm can be invoked.
The influence of the next-to-leading-order corrections, along with other
options regarding electron-positron pair, muon pair and photon emission, can be studied. \\
   \\
{\em Solution method:}\\
  %Describe the method solution here.
The event record is traversed and a list of all decaying particles is created.
Decays where program is not supposed to act and decays excluded by the user
are removed from the list. The photon and pair adding algorithm is invoked
separately on each remaining decay. If one or more particle is added to the
decay, the kinematic of the whole decay tree is updated. \\
  \\
{\em Restrictions:}\\
  %Describe any restrictions on the complexity of the problem here.
Application of the algorithm strongly depends on the content on the event
record. Insufficient precision or missing information may deteriorate
quality of the results of the algorithm or prevent algorithm from
working on the event or its parts. See Section \ref{sec:requrements} for more details. \\
   \\
{\em Unusual features:}\\
  %Describe any unusual features of the program/problem here.
   \\
{\em Additional comments:}\\
  %Provide any additional comments here.
   \\
{\em Running time:}\\
  %Give an indication of the typical running time here.
10-30 seconds per 100k events for small events (less than 1k particles).
The complexity strongly depends on the event content and user selection
of excluded decays. The theoretical pessimistic complexity of the algorithm is $O(n^2)$.
However, such cases are highly unrealistic.
In our tests, the average complexity is around $O(n^{1.2})$. \\
   \\
\end{small}

